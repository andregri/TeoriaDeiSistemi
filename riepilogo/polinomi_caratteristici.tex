\documentclass[../main.tex]{subfiles}

\begin{document}
	\section{Polinomi caratteristici}
		\bgroup
		\def\arraystretch{0.2} %  serve per impostare l'interlinea
		\begin{tabular}{|>{\centering\arraybackslash}m{1.5in}|>{\centering\arraybackslash}m{1in}|>{\centering\arraybackslash}m{1in}|>{\centering\arraybackslash}m{1.5in}|}
			\hline
			$$ MATRICE $$ & $$ POLINOMIO $$ & $$ GRADO $$ & $$ INFORMAZIONI $$\\
			\hline
			$$ (sI-A)^{-1} $$ & $$ \varphi(s) $$ & $$ n_x $$ & $$ \text{stabilit\'a interna}\ x_l(t) $$ \\
			\hline
			$$ (sI-A)^{-1}B $$ & $$ \varphi_C(s) $$ & $$ n_R $$ & $$ \text{controllabilit\'a}\ x_f(t) $$ \\
			\hline
			$$ C(sI-A)^{-1} $$ & $$ \varphi_O(s) $$ & $$ n_O $$ & $$ \text{osservabilit\'a}\ y_l(t) $$ \\
			\hline
			$$ C(sI-A)^{-1}B $$ & $$ \varphi_{CO}(s) $$ & $$ n_{CO} $$ & $$ \text{stabilit\'a BIBO}\ y_f(t) $$ \\
			\hline
		\end{tabular}
		\egroup
\end{document}