\documentclass[../main.tex]{subfiles}

\begin{document}
	\section{Polinomi caratteristici}
		\bgroup
		\def\arraystretch{0.2} %  serve per impostare l'interlinea
		\begin{tabular}{|>{\centering\arraybackslash}m{1.5in}|>{\centering\arraybackslash}m{1in}|>{\centering\arraybackslash}m{1in}|>{\centering\arraybackslash}m{1.5in}|}
			\hline
			$$ MATRICE $$ & $$ POLINOMIO $$ & $$ GRADO $$ & $$ INFORMAZIONI $$\\
			\hline
			$$ (sI-A)^{-1} $$ & $$ \varphi(s) $$ & $$ n_x $$ & $$ \text{stabilit\'a interna}\ x_l(t) $$ \\
			\hline
			$$ (sI-A)^{-1}B $$ & $$ \varphi_C(s) $$ & $$ n_R $$ & $$ \text{controllabilit\'a}\ x_f(t) $$ \\
			\hline
			$$ C(sI-A)^{-1} $$ & $$ \varphi_O(s) $$ & $$ n_O $$ & $$ \text{osservabilit\'a}\ y_l(t) $$ \\
			\hline
			$$ C(sI-A)^{-1}B $$ & $$ \varphi_{CO}(s) $$ & $$ n_{CO} $$ & $$ \text{stabilit\'a BIBO}\ y_f(t) $$ \\
			\hline
		\end{tabular}
		\egroup
		
	\section{Stabilit\'a BIBO}
		Un sistema \'{e} stabile BIBO se e solo se $ \nexists $ poli di $ T(s) $ (radici del denominatore di $ T(s) $) a $ \Re > 0 $.
		
	\section{Stabilit\'a}
		\begin{itemize}
			\item
				\textbf{asintoticamente stabile} se tutte le radici di $ m(s) $ hanno $ \Re < 0 $;
			\item
				\textbf{instabile} se $ m(s) $ ha almeno una radice a $ \Re > 0 $ oppure esiste almeno una radice a $ \Re = 0 $ ma tutte con molteplicit\'a $ >1 $ in $ m(s) $;
			\item
				\textbf{semplicemente stabile} in tutte le altre situazioni: nessuna radice a $ \Re > 0 $ e esiste almeno una radice a $ \Re = 0 $ con molteplicit\'a $ =1 $ in $ m(s) $.
		\end{itemize}
	
	\section{Relazioni tra le due stabilit\'a}
		\[
			\begin{aligned}
				\text{asintoticamente stabile} &\quad\Rightarrow\quad \text{stabile BIBO}
				\\
				\text{NON stabile BIBO} &\quad\Rightarrow\quad \text{NON asintoticamente stabile}
			\end{aligned}
		\]
\end{document}