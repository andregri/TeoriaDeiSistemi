% !TeX spellcheck = it_IT
\documentclass[../main.tex]{subfiles}

\begin{document}
	\section{Metodi utili di calcolo matriciale}
	%
	\subsection{Inversione di matrici $3 \times 3$}
	\label{inv_matrici}
	Presa una matrice $A$ di dimensione $3 \times 3$, consideriamo l'elemento in posizione $(i,j)$ chiamato $a_{ij}$ e calcoliamo il corrispettivo valore $a'_{ij}$, l'elemento in posizione $(i,j)$ della matrice inversa di $A$.
	%	
			\[	A=
	\begin{tikzpicture}[baseline=-0.5ex]
	\matrix [matrix of math nodes,left delimiter={[},right delimiter={]}] (m1)
	{
		a_{11} & a_{12}  & a_{13}\\
		a_{21} & a_{22}  & a_{23}\\
		a_{31} & a_{32}  & a_{33}\\
	};  
	\draw[color=red] (m1-2-1.west) -- (m1-2-3.east);
	\draw[color=red] (m1-1-1.north) -- (m1-3-1.south);
	\end{tikzpicture} 
	%
	\;\;\; A^{-1}=
	\begin{tikzpicture} [baseline=-0.5ex]
	\matrix [matrix of math nodes,left delimiter={[},right delimiter={]}] (m2)
	{
		a'_{11} & a'_{12}  & a'_{13}\\
		a'_{21} & a'_{22}  & a'_{23}\\
		a'_{31} & a'_{32}  & a'_{33}\\
	};  
	\end{tikzpicture} 
	%
	\;\;\; M=
	\begin{tikzpicture} [baseline=-0.5ex]
	\matrix [matrix of math nodes,left delimiter={[},right delimiter={]}] (m3)
	{
		a'_{12}  & a'_{13}\\
		a'_{32}  & a'_{33}\\
	};  
	\end{tikzpicture} 
	%
	\;\;\; 	a'_{12}= (-1)^{i+j}	\frac{\det{M}}{\det{A}}
	\]
	%
	Consideriamo l'elemento nella "posizione trasposta" $(j,i)$ chiamato $a_{ji}$\\
	Cancelliamo la sua riga e la sua colonna e consideriamo i quattro coefficienti rimanenti\\
	Calcoliamo il determinante del minore $M$ $2 \times 2$ cos\'i ricavato\\
	Dividiamo il risultato per il determinante di $A$ e moltiplichiamolo per $(-1)^{i+j}$
	%
	\subsection{Inversione di matrici triangolari a blocchi}
	L'inversa di una matrice triangolare superiore invertibile \'e sempre una matrice triangolare superiore. Questa propriet\'a \'e estensibile alle matrici triangolari a blocchi $A_{ij}$.
	Avremo perci\'o nell'inversa tutti i blocchi $A'_{ij}$ al di sotto della diagonale composti anch'essi da zeri. I blocchi della diagonale potranno essere calcolati invertendo ogni singolo blocco. I blocchi restanti (?) andranno calcolati "normalmente", con il metodo illustrato in \ref{inv_matrici}. 
				\[ 	A= 			
	\begin{tikzpicture}[baseline=-0.5ex]
	\matrix [matrix of math nodes,left delimiter={[},right delimiter={]}] (a)
	{
		A_{11}	& A_{12}	& ...		& A_{1m}\\
		0		& A_{22}	& ...		& A_{2m}\\
		0		& 0			& ...		& ...	\\
		0		& 0			& 0			& A_{nm}\\
	};  
	\draw[color=red] (a-2-1.north west) -- (a-4-3.south east) -- (a-4-1.south west) -- (a-2-1.north west);
	\draw[color=green] (a-1-1.north west) -- (a-4-4.south east);
	\end{tikzpicture} 
	%
	\;\;\; A^{-1}=
	\begin{tikzpicture}[baseline=-0.5ex]
	\matrix [matrix of math nodes,left delimiter={[},right delimiter={]}] (b)
	{
		(A_{11})^{-1}	& (?)			& ...		& (?)		\\
		0				& (A_{22})^{-1}	& ...		& (?)		\\
		0				& 0				& ...		& ...			\\
		0				& 0				& 0			& (A_{nm})^{-1}	\\
	};  
	\draw[color=red] (b-2-1.north west) -- (b-4-3.south east) -- (b-4-1.south west) -- (b-2-1.north west);
	\draw[color=green] (b-1-1.north west) -- (b-4-4.south east);
	\end{tikzpicture} 
	\]
\end{document}