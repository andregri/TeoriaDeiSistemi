\documentclass[../main.tex]{subfiles}

\begin{document}
	Il sistema $ S $ \'e costituito da $ N $ sottosistemi $ S^{(i)} $ con $ i = 1...N $, ognuno con il suo vettore $ \vec x^{(i)} $.
	\[
		\begin{aligned}
			&\text{vettore degli ingressi esogeni}
			\\
			&\vec u =
			\begin{bmatrix}
				u_1\\
				\vdots\\
				u_{n_u}
			\end{bmatrix}
		\end{aligned}
		\qquad\qquad
		\begin{aligned}
			&\text{vettore degli uscite}
			\\
			&\vec y =
			\begin{bmatrix}
				y_1\\
				\vdots\\
				y_{n_y}
			\end{bmatrix}
		\end{aligned}
	\]
	\[
		\begin{aligned}
			&\text{vettore degli ingressi di $ S^{(i)} $}
			\\
			&\vec u =
			\begin{bmatrix}
				u_1\\
				\vdots\\
				u_{n_u}
			\end{bmatrix}
		\end{aligned}
		\qquad\qquad
		\begin{aligned}
			&\text{vettore degli uscite di $ S^{(i)} $}
			\\
			&\vec y =
			\begin{bmatrix}
				y_1\\
				\vdots\\
				y_{n_y}
			\end{bmatrix}
		\end{aligned}
	\]
	
	\section{Studio delle propriet\'a strutturali}
		\begin{enumerate}
			\item 
				Mettere in equazione di stato ogni singolo sistema $ S^{(i)} $ con realizzazione minima
				\[
					S^{(i)}:
					\begin{cases}
						\dot{\vec x}^{(i)} = A^{(i)} \vec x^{(i)} + B^{(i)} \vec u^{(i)}\\
						\dot{\vec y}^{(i)} = C^{(i)} \vec x^{(i)} + D^{(i)} \vec u^{(i)}
					\end{cases}
				\]
			\item 
				Scrivo le equazioni di interconnessione, cio\'e esprimo $ \vec u^{(i)} $ e $ \vec y $ come funzione (combinazione lineare) di tutte le uscite $ \vec y^{(i)} $ e di tutti gli ingressi esogeni $ \vec u $.
			\item 
				Per sostituzioni successive, elimino le dipendenze da $ \vec u^{(i)} $ e da $ \vec y^{(i)} $. Otteniamo un sistema lineare:
				\[
					\begin{cases}
						\dot{\vec x} = A \vec x + B \vec u\\
						\vec y = C \vec x + D \vec u
					\end{cases}
				\]
				NB: non devono essere presenti loop algebrici, altrimenti questo passaggio non ha fine.
			\item 
				Studio delle propriet\'a strutturali (Controllabilit\'a, Osservabilit\'a, Stabilit\'a) del sistema di equazioni di stato cos\'i ottenuto
		\end{enumerate}
	
		\paragraph{Esempio}
			\[
				\begin{aligned}
					T^{(1)}(s) &= \dfrac{1}{s+1}\\
					\varphi_{CO}^{(1)}(s) &= s+1
				\end{aligned}
				\qquad\qquad
				\begin{aligned}
					T^{(2)}(s) &= \dfrac{s+1}{s}\\
					\varphi_{CO}^{(2)}(s) &= s
				\end{aligned}
			\]
			Si tratta di un sistema di ordine 2. Eseguiamo tutti i passaggi:
			\begin{enumerate}
				\item 
					\[
						S^{(1)}:
						\begin{cases}
							\dot x^{(1)} = -x^{(1)} + u^{(1)}\\
							y^{(1)} = x^{(1)}
						\end{cases}
						\qquad
						S^{(2)}.
						\begin{cases}
							\dot x^{(2)} = u^{(2)}\\
							y^{(2)} = x^{(2)} + u^{(2)}
						\end{cases}
					\]
				\item 
					\[
						u^{(1)} = u \qquad u^{(2)} = y^{(1)} \qquad y = y^{(2)}
					\]
				\item 
					\[
						\begin{aligned}
							\dot x^{(1)} &= -x^{(1)} + u\\
							y^{(1)} &= x^{(1)}\\
							\dot x^{(2)} &= y^{(1)} = x^{(1)}\\
							y^{(2)} &= x^{(2)} + y^{(1)} = x^{(2)} + x^{(1)}\\
							y &= y^{(2)} = x^{(2)} + u^{(2)} = x^{(2)} + y^{(1)} = x^{(2)} + x^{(1)}
						\end{aligned}
					\]
					\[
						\Rightarrow\quad S:
						\begin{cases}
							\dot{\vec x} =
							\begin{bmatrix}
								-1 & 0\\
								1 & 0
							\end{bmatrix} \vec x +
							\begin{bmatrix}
								1\\
								0
							\end{bmatrix} u
							\\
							y =
							\begin{bmatrix}
								1 & 1
							\end{bmatrix} \vec x
						\end{cases}
					\]
				\item 
					\[
						\varphi(s) = det(sI-A) = det
						\begin{bmatrix}
							s+1 & 0\\
							-1 & s
						\end{bmatrix} = s(s+1)
						\qquad
						m(s) \equiv \varphi(s)\ \text{perch\'e}\ m(s) \subseteq\cdot \varphi(s)
					\]
					Il sistema \'e semplicemente stabile perch\'e c'\'e un autovalore a $ \Re = 0 $ e uno a $ \Re < 0 $
					\[
						P =
						\begin{bmatrix}
							1 & | & -1\\
							0 & | & 1
						\end{bmatrix}
						\quad\Rightarrow\quad
						rank(P) = 2
					\]
					Il sistema \'e completamente controllabile.
					\[
						Q =
						\begin{bmatrix}
							1 & 1\\
							\cline{1-2}
							0 & 0
						\end{bmatrix}
						\quad\Rightarrow\quad
						rank(Q) = 1
					\]
					Il sistema non \'e completamente osservabile.
					\[
						(sI-A)^{-1} =
						\begin{bmatrix}
							\dfrac{1}{s+1} & 0\\
							\dfrac{1}{s(s+1)} & \dfrac{1}{s}
						\end{bmatrix}
					\]
					\[
						(sI-A)^{-1}B =
						\begin{bmatrix}
							\dfrac{1}{s+1}\\
							\dfrac{1}{s(s+1)}
						\end{bmatrix}
						\quad\Rightarrow\quad
						\varphi_C(s) = s(s+1)
					\]
					\[
						C(sI-A)^{-1} = 
						\begin{bmatrix}
							\dfrac{1}{s+1}+\dfrac{1}{s(s+1)} & \dfrac{1}{s}
						\end{bmatrix} =
						\begin{bmatrix}
							\dfrac{1}{s} & \dfrac{1}{s}
						\end{bmatrix}
						\quad\Rightarrow\quad
						\varphi_O(s) = s
					\]
			\end{enumerate}
		
			\begin{Exercise}[title={Studiare propriet\'a di sistemi interconnessi}, difficulty=3]
				Sono dati i sistemi:
				\[
					S^{(1)}:
					\begin{cases}
						\dot{\vec x}^{(1)} =
						\begin{bmatrix}
							0 & 1\\
							-1 & 2
						\end{bmatrix}
						\vec x^{(1)} +
						\begin{bmatrix}
							0\\
							1
						\end{bmatrix} u^{(1)}
						\\
						y^{(1)} =
						\begin{bmatrix}
							1 & 1\\
							0 & 1
						\end{bmatrix} \vec x^{(1)}
					\end{cases}
				\]
				\[
					S^{(2)}: T(s)^{(2)}=\dfrac{1}{s+1}
				\]
				Sono due sistemi strettamente propri quindi non ci sono loop algebrici. Metto in equazione di stato il 2° sistema:
				\[
					S^{(2)}:
					\begin{cases}
						\dot x^{(2)} = -x^{(2)} + u^{(2)}\\
						y^{(2)} = x^{(2)}
					\end{cases}
				\]
				\begin{enumerate}
					\item 
						\begin{align*}
							\dot x^{(1)}_1 = x^{(1)}_2\\
							\dot x^{(1)}_2 = -x^{(1)}_1 + 2x^{(1)}_2 + u^{(1)}\\
							y^{(1)}_1 = x^{(1)}_1 + x^{(1)}_2\\
							y^{(1)}_2 = x^{(1)}_2
						\end{align*}
					\item 
						\begin{align*}
							u^{(1)} = y^{(2)} + u\\
							u^{(2)} = y^{(1)}_1\\
							y_1 = y^{(1)}_1\\
							y_2 = y^{(1)}_2\\
							y_3 = y^{(2)}
						\end{align*}
					\item 
						\begin{align*}
							\dot x^{(1)}_2 = -x^{(1)}_1 + 2x^{(1)}_2 + y^{(2)} + u = -x^{(1)}_1 + 2x^{(1)}_2 + x^{(2)} + u\\
							\dot x^{(2)} = -x^{(2)} + y^{(1)}_1 = -x^{(2)} + x^{(1)}_1 + x^{(1)}_2\\
							y_1 = x^{(1)}_1 + x^{(1)}_2\\
							y_2 = x^{(1)}_2\\
							y_3 = x^{(2)}
						\end{align*}
						\[
							\begin{aligned}
								\dot{\vec x} = 
								\begin{bmatrix}
									0 & 1 & 0\\
									-1 & 2 & 1\\
									1 & 1 & -1
								\end{bmatrix} \vec x+
								\begin{bmatrix}
									0\\
									1\\
									0
								\end{bmatrix} u
								\\
								\vec y = 
								\begin{bmatrix}
									1 & 1 & 0\\
									0 & 1 & 0\\
									0 & 0 & 1
								\end{bmatrix} \vec x +
								\begin{bmatrix}
									0\\
									0\\
									0
								\end{bmatrix} u
							\end{aligned}
						\]
					\item 
						\begin{itemize}
							\item Controllabilit\'a:
								\[
									P=
									\begin{bmatrix}
										0 &|& 1 &|& 2\\
										1 &|& 2 &|& 4\\
										0 &|& 1 &|& 2
									\end{bmatrix}
									\quad\Rightarrow\quad
									rank(P) =2
								\]
							\item Osservabilit\'a:
								\[
									Q=
									\begin{bmatrix}
										1 & 1 & 0\\
										0 & 1 & 0\\
										0 & 0 & 1\\
										\cline{1-3}\\
										&\dots&
									\end{bmatrix}
								\]
								Poich\'e gi\'a $ rank(C) = 3 $, il sistema \'e completamente osservabile.
								
								Calcoliamo completamente $ Q $ per studiare l'osservabilit\'a da ogni singola uscita:
								\[
									Q=
									\begin{bmatrix}
										1 & 1 & 0\\
										0 & 1 & 0\\
										0 & 0 & 1\\
										\cline{1-3}\\
										-1 & 3 & 1\\
										-1 & 2 & 1\\
										1 & 1 & -1\\
										\cline{1-3}\\
										-2 & 6 & 2\\
										-1 & 4 & 1\\
										-2 & 2 & 2
									\end{bmatrix}
								\]
								Le prime righe di ogni sottomatrice non sono linearmente indipendenti, quindi il sistema non \'e completamente osservabile da $ y_1 $.\\
								Le seconde righe di ogni sottomatrice non sono linearmente indipendenti, quindi il sistema non \'e completamente osservabile da $ y_2 $.\\
								Le terze righe di ogni sottomatrice sono linearmente indipendenti(bisogna calcolare tutti i minori), quindi il sistema non \'e completamente osservabile da $ y_3 $.\\
							\item Stabilit\'a:
								\[
									\varphi(s) = det
									\begin{bmatrix}
										s & -1 & 0\\
										1 & s-2 & -1\\
										-1 & -1 & s+1
									\end{bmatrix} =
									(-s-1)+(s+1)(s^2-2s+1) = (s+1)(s^2-2s+1-1) =(s+1)s(s-2)
								\]
								poich\'e $ m(s) \subseteq\cdot \varphi(s) $, allora $ m(s) \equiv \varphi(s) $. Il sistema \'e instabile.
							\item Stabilit\'a BIBO:\\
								se nella matrice di trasferimento scompaiono gli autovalori $ 0 $ e $ +2 $, sarebbe stabile BIBO, cio\'e se $ \varphi_{CO}(s) = \dfrac{1}{s+1} $ o $ \varphi_{CO}(s) = 1 $. Ma abbiamo trovato che $ \pDeg{\varphi_C(s)} = 2 $ e $ \pDeg{\varphi_{O}(s)} = 3 $, quindi $ \pDeg{\varphi_{CO}(s)} = 2 $ (\'e l'intersezione dei gradi $ \varphi_C(s) $ e $ \varphi_O(s) $), cio\'e conterr\'a almeno uno di quei due autovalori ($ 0 $ o $ +2 $). Il sistema \'e instabile BIBO.
						\end{itemize}
				\end{enumerate}
			\end{Exercise}
		
		Molte delle propriet\'a strutturali possono essere desunte sfruttando relazioni note tra i polinomi.
		\paragraph{Esempio}
			\[
				T(s) = \dfrac{1}{s^2}
			\]
			\begin{itemize}
				\item Stabilit\'a BIBO:\\
					non \'e stabile perch\'e c'\'e una radice a $ \Re = 0 $
				\item Stabilit\'a:\\
					\[
						\varphi_{CO}(s) = s^2 \equiv m_{CO}(s)
					\]
					\begin{itemize}
						\item 
							$ \varphi_{CO}(s) \subseteq \varphi(s) $ ($ s^2 \subseteq \varphi(s) $) allora $ \varphi(s) $ avr\'a una radice a $ \Re = 0 $. Il sistema non pu\'o essere asintoticamente stabile.
						\item 
							$ m_{CO}(s) \subseteq m(s) $ ($ s^2 \subseteq m(s) $) allora $ m(s) $ ha una radice a $ \Re = 0 $ con molteplicit\'a 2. Il sistema \'e instabile.
					\end{itemize}
			\end{itemize}
		
	\section{Sistemi serie/parallelo}
		Il polinomio caratteristico di polinomi connessi unicamente serie e/o parallelo \'e dato dal prodotto dei singoli polinomi caratteristici.
		\paragraph{Dimostrazione serie}
			\[
				S^{(1)}:
				\begin{cases}
					\dot x^{(1)} = A^{(1)} x^{(1)} + B^{(1)} u^{(1)}\\
					y^{(1)} = C^{(1)} x^{(1)} + D^{(1)} u^{(1)}
				\end{cases}
				\qquad
				S^{(2)}:
				\begin{cases}
					\dot x^{(2)} = A^{(2)} x^{(2)} + B^{(2)} u^{(2)}\\
					y^{(2)} = C^{(2)} x^{(2)} + D^{(2)} u^{(2)}
				\end{cases}
			\]
			Le equazioni di interconnessione sono:
			\[
				u^{(2)} = y^{(1)} \qquad u^{(1)} = u \qquad y = y^{(2)}
			\]
			\[
				\begin{aligned}
					\dot x^{(2)} = A^{(2)} x^{(2)} + B^{(2)} \left[ C^{(1)} x^{(1)} + D^{(1)} u \right] = A^{(2)} x^{(2)} + B^{(2)} c^{(1)} x^{(1)} + B^{(2)} D^{(1)} u
					\\
					y^{(2)} = C^{(2)} x^{(2)} + D^{(2)} C^{(1)} x^{(1)} + D^{(2)} D^{(1)} u
				\end{aligned}
			\]
			\[
				\begin{cases}
						\dot{\vec x} =
						\begin{bmatrix}
							A^{(1)} & 0\\
							B^{(2)} C^{(1)} & A^{(2)}
						\end{bmatrix} \vec x +
						\begin{bmatrix}
							B^{(1)}\\
							B^{(2)} D^{(1)}
						\end{bmatrix} \vec u
						\\
						\vec y =
						\begin{bmatrix}
							D^{(2)} C^{(2)} & C^{(2)}
						\end{bmatrix} \vec x + 
						D^{(2)} D^{(1)} \vec u
				\end{cases}
			\]
			$ A_{serie} $ \'e una matrice triangolare a blocchi quindi:
			\[
				\varphi_{serie}(s) = det(sI-A_{serie}) = det(sI-A^{(1)}) \cdot det(sI-A^{(2)}) = \varphi^{(1)}(s) \cdot \varphi^{(2)}(s)
			\]
		\paragraph{Parallelo}
			\[
				\dot{\vec x} = 
				\begin{bmatrix}
					A^{(1)} & 0\\
					0 & A^{(2)}
				\end{bmatrix} \vec x +
				\begin{bmatrix}
					B^{(1)}\\
					B^{(2)}
				\end{bmatrix} u
				\\
				y = 
				\begin{bmatrix}
					C^{(1)} & C^{(2)}
				\end{bmatrix} \vec x +
				\left( D^{(1)} + D^{(2)} \right) u
			\]
			$ A_{parallelo} $ \'e una matrice diagonale:
			\[
				det(sI-A_{parallelo}) = det(sI-A^{(1)}) \cdot det(sI-A^{(2)}) = \varphi_{parallelo}(s) = \varphi^{(1)}(s) \cdot \varphi^{(2)}(s)
			\]
	\subsection{Corollario}
		Condizione necessaria e sufficiente per asintotica stabilit\'a di un sistema serie e/o parallelo \'e l'asintotica stabilit\'a dei singoli sistemi. Non vale per la semplice stabilit\'a.
\end{document}