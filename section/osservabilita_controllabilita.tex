\documentclass[../main.tex]{subfiles}

\begin{document}
	\section{Osservabilit\'a e ricostruibilit\'a}
		\[
			S:
			\begin{cases}
				\dot x &= Ax + Bu\\
				y &= Cx + Du
			\end{cases}
		\]
		Potremmo essere interessati a conoscere all'istante $ t $, lo stato $ x(t) $ o la condizione iniziale $ x(0^{-}) $, perch\'e $ x(t) $ pu\'o avere un significato fisico. Potremmo non averne una misura diretta. Inoltre posso voler utilizzare lo stato $ x(t) $ per esempio per fare feedback. Spesso (nella pratica sempre) ci si accontenta di una stima $ \hat x(t) $ dello stato.
		
		\begin{definition}
			Un sistema \'e completamente osservabile \textbf{se e solo se}, dati $ u(\tau) $ e $ y(\tau) $ per $ \tau \in [0,t] $, \'e possibile calcolare univocamente $ x(0^{-}) $.
		\end{definition}
	
		\begin{definition}
			Un sistema \'e completamente ricostruibile \textbf{se e solo se}, dati $ y(\tau) $ e $ u(\tau) $ per $ \tau \in [0,t] $, \'e possibile calcolare univocamente $ x(t) $.
		\end{definition}
	
		Poich\'e $ x(t) $ \'e sempre funzione della condizione iniziale $ x(0^{-}) $ e della sequenza di controllo $ u(\tau) $ con $ \tau \in [0,t] $, se un sistema \'e osservabile allora sar\'a anche ricostruibile. Non vale necessariamente il viceversa.
		
		Per i sistemi LTI a tempo continuo valgono entrambe le implicazioni:
		\begin{center}
			un sistema \'e osservabile $ \Leftrightarrow $ \'e ricostruibile.
		\end{center}
		Dimostrazione:\\
		partiamo dall'equazione di Lagrange da cui \'e evidente che $ x(t) $ dipenda da $ x(0^{-}), u(.) $:
		\[
			x(t) = e^{At}x(0^{-}) + \int_{0^{-}}^{t} e^{A(t-\tau)} B\ u(\tau) d\tau
		\]
		ma poich\'e $ e^{At} $ \'e sempre invertibile ($ (e^{At})^{-1} = e^{-At} $) otteniamo anche che la condizione iniziale $ x(0^{-}) $ dipende da $ x(t) $ e $ u(.) $:
		\begin{align*}
			e^{At} x(0^{-}) &= x(t) - \int_{0^{-}}^{t} e^{A(t-\tau)} B\ u(\tau) d\tau\\
			x(0^{-}) &= e^{-At} \left[ x(t) - \int_{0^{-}}^{t} e^{A(t-\tau)} B\ u(\tau) d\tau \right] 
		\end{align*}
\end{document}