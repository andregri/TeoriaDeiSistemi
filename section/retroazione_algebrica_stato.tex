\documentclass[../main.tex]{subfiles}

\begin{document}
	\section{Retroazione algebrica sullo stato}
		Dato uno sistema:
		\[
			S: 
			\begin{cases}
				\dot x &= Ax + Bu\\
				y &= Cx + Du
			\end{cases}
		\]
		\'E possibile modificare gli autovalori della matrice $ A $ dello stato, attraverso una retroazione $ u = Kx + v $ dove $ K \in \R^{n_u \times n_x} $:
		\[
			S^{*}: \footnotemark
			\begin{cases}
				\dot x &= Ax + B(Kx+v) = (A+BK)x + Bv = A^{*}x + Bv\\
				y &= Cx +D(Kx+v) = (C+DK)x + Dv = 
			\end{cases}
		\]
		\footnotetext{
			Notazione:
			\begin{itemize}
				\item $ \tilde{} $ indica le matrici relative al sistema ottenuto tramite la decomposizione di Kalmann di raggiungibilit\'a.
				\item $ * $ indica le matrici relative a un sistema con la retroazione sullo stato.
			\end{itemize}}
		Effettuiamo la decomposizione di Kalmann di raggiungibilit\'a:
		\[ x = Tz \qquad T = [T_1 | T_2] \]
		\[
			\tilde S:
			\begin{cases}
				\dot z &= \tilde Az + \tilde Bu\\
				y &= \tilde Cz + \tilde Du
			\end{cases}
		\]
		La retroazione nella nuova base diventa:
		\[ u = KTz + v = \tilde Kz + v \]
		Quindi il sistema decomposto con la retroazione sullo stato \'e:
		\[
			\tilde{S^{*}}:
			\begin{cases}
				\dot z &= (\tilde A + \tilde B \tilde K)z + \tilde Bv = \tilde{A^{*}}z + \tilde{B} v\\
				y &= \tilde{C^{*}}z + \tilde Du
			\end{cases}
		\]
		\begin{align*}
			\tilde{A^{*}} = \tilde A + \tilde B \tilde K &=
			\begin{bmatrix}
				\tilde A_{11} & \tilde A_{12}\\
				0 & \tilde A_{22}
			\end{bmatrix} +
			\begin{bmatrix}
				\tilde B_{1}\\
				0
			\end{bmatrix} K
			\begin{bmatrix}
				T_1 | T_2
			\end{bmatrix}=
			\\
			&= \begin{bmatrix}
				\tilde A_{11} & \tilde A_{12}\\
				0 & \tilde A_{22}
			\end{bmatrix} +
			\begin{bmatrix}
				\tilde B_1 K T_1 & \tilde B_1 K T_2\\
				0 & 0
			\end{bmatrix}=
			\\
			&= \begin{bmatrix}
				\tilde A_{11} + \tilde B_1 K T_1 & \tilde A_{12} + \tilde B_1 K T_2\\
				0 & \tilde A_{22}
			\end{bmatrix}
		\end{align*}
		Il polinomio caratteristico diventa:
		\begin{align*}
			\varphi^{*} &= det(sI-A^{*}) = \tilde{\varphi^{*}} = det(sI-\tilde{A^{*}}) =\\
			&= det\left[ sI - (\tilde A_{11} + \tilde B_1 K T_1) \right] \cdot \underbrace{det(sI - \tilde A_{22})}_{\varphi_{NC}}
		\end{align*}
		
		\paragraph{Osservazioni}
		\begin{itemize}
			\item
				Ho verificato che la retroazione algebrica sullo stato non modifica $ \varphi_{NC}(s) $;
			\item
				$ \varphi_C(s) $ viene modificata dalla retroazione algebrica su $ x $;
			\item
				$ \varphi_C(s) $ pu\'o essere \textbf{assegnato a piacere} con una retroazione algebrica sullo stato $ x $, cio\'e \'e possibile scegliere $ K $ tale che $ det\left[ sI - (\tilde A_{11} + \tilde B_1 K T_1) \right] $ abbia autovalori arbitrari.
		\end{itemize}
	
		
		\paragraph{Definizioni}
		\begin{center}
			\begin{minipage}{.9\linewidth}
				\textbf{Autovalori controllabili} sono tutti e soli gli autovalori assegnabili a piacere con retroazione algebrica sullo stato.
			\end{minipage}
		\end{center}
	
		\begin{center}
			\begin{minipage}{.9\linewidth}
				Un sistema \'e \textbf{stabilizzabile} (cio\'e \'e possibile renderlo asintoticamente stabile) se e solo se la sua parte non controllabile \'e asintoticamente stabile ($ \Leftrightarrow $ gli autovalori di $ \varphi_{NC}(s) $ sono tutti a $ \Re < 0 $).
			\end{minipage}
		\end{center}
\end{document}