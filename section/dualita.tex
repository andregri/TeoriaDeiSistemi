\documentclass[../main.tex]{subfiles}

\begin{document}
	\section{Dualit\'a}
		Prendiamo due sistemi diversi:
		\[
			S:
			\begin{cases}
				\dot x = Ax + Bu\\
				y = Cx
			\end{cases}
			\qquad
			\hat S:
			\begin{cases}
				\dot{\hat x} = \hat A \hat x + \hat B \hat u\\
				\hat y = \hat C \hat x
			\end{cases}
		\]
		dove:
		\[
			\hat A = A^T \qquad \hat B = C^T \qquad \hat C = B^T
		\]
		\[
			\Rightarrow\quad n_{\hat y} = n_u \qquad n_{\hat u} = n_y \qquad n_{\hat x} = n_x
		\]
		
		Due sistemi sono \textbf{duali} se le propriet\'a di controllabilit\'a/osservabilit\'a di $ \hat S $ corrispondono \textbf{rispettivamente} a quelle di osservabilit\'a/controllabilit\'a di $ S $.
		
		Si dimostra con il teorema di Kalman:
		\[
			\hat P =
			\begin{bmatrix}
				\hat A & | & \hat A \hat B & | & \dots & | & \hat{A}^{n_x-1} \hat B
			\end{bmatrix} =
			\begin{bmatrix}
				C^T &|& A^T C^T &|& \dots &|& \underbrace{A^T \dots A^T}_{n_x-1 \text{volte}} C^T
			\end{bmatrix} =
			\begin{bmatrix}
				C\\
				CA\\
				\vdots\\
				CA^{n_x-1}
			\end{bmatrix} = Q^T
		\]
		Analogamente si ottiene che $ \hat Q = P^T $.\\
		\newline
		Le matrici $ P $ e $ Q $ vengono solamente trasposte, quindi i ranghi non vengono modificati.\\
		Poich\'e $ \hat A = A^T $, gli autovalori non cambiano: $ \varphi(s) = \hat \varphi(s) $. Inoltre per la relazione che lega $ \hat B $ con $ \hat C $:
		\[
			\varphi_C(s) = \hat \varphi_O(s) \qquad \varphi_O(s) = \hat \varphi_C(s) \qquad \varphi_{CO}(s) = \hat \varphi_{CO}(s)
		\]
		
		La matrice di trasferimento:
		\begin{align*}
			\hat T(s) &= \hat C(sI-\hat A)^{-1} \hat B = B^T (sI-A^T)^{-1} C^T =\\
			&= \left[ B^T (sI-A)^{-1} \right]^{T} C^T = \left[ C(sI-A)^{-1} B \right]^T = T^T(s) 
		\end{align*}
\end{document}