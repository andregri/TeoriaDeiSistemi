\documentclass[../main.tex]{subfiles}

\begin{document}
	\section{Realizzazione}
		Si tratta del problema di trovare, data $ T(s) $, $ A, B, C, D $ tali che $ T(s) = C(sI-A)^{-1}B + D $.
		
		Il problema della realizzazione non ha soluzione unica per i seguenti motivi:
		\begin{itemize}
			\item 
				possibili cambi di base: se $ A, B, C, D $ sono una realizzazione di $ T(s) $, allora anche $ T^{-1}AT, T^{-1}B, CT, D $ lo sono $ \forall T \in \R^{n_x \times n_x} $ invertibile.
			\item 
				possibile aggiunta di parti non controllabili o non osservabili, dato che la funzione di trasferimento \'e influenzata solo dalla parte controllabile e osservabile.
		\end{itemize}
	
		\begin{mdframed}[style=Esempio]
			\paragraph{Esempio}
			\[
				T(s) = \dfrac{1}{s^2}				
			\]
			\'e un sistema strettamente proprio perch\'e $ \pDeg{\text{num}} < \pDeg{\text{denom}} \quad\Rightarrow\quad D = 0 $ (non c'\'e dipendenza istantanea tra ingresso e uscita).
			
			Consideriamo diversi sistemi e verifichiamo se sono realizzazioni di $ T(s) $:
			\begin{itemize}
				\item 
					\[
						S:
						\begin{cases}
							\dot x =
							\begin{bmatrix}
								0 & 1\\
								0 & 0
							\end{bmatrix} x+
							\begin{bmatrix}
								0\\
								1
							\end{bmatrix} u
							\\[1em]
							y =
							\begin{bmatrix}
								1 & 0
							\end{bmatrix} x
						\end{cases}
					\]
					Il sistema $ S $ \'e una realizzazione di $ T(s) $? Verifichiamolo:
					\[
						\begin{bmatrix}
							1 & 0
						\end{bmatrix}
						\begin{bmatrix}
							\dfrac{1}{s} & \dfrac{1}{s^2}
							\\[1em]
							0 & \dfrac{1}{s}
						\end{bmatrix}
						\begin{bmatrix}
							0\\
							1
						\end{bmatrix} = \dfrac{1}{s^2}
					\]
				\item 
					effettuiamo un cambio di base attraverso la matrice $ T = \begin{bmatrix} 0 & 1\\ 1 & 0 \end{bmatrix} $, che inverte il ruolo degli stati:
					\[
						\begin{cases}
						\dot z =
						\begin{bmatrix}
							0 & 0\\
							1 & 0
						\end{bmatrix} z+
						\begin{bmatrix}
							1\\
							0
						\end{bmatrix} u
						\\[1em]
						y =
						\begin{bmatrix}
							0 & 1
						\end{bmatrix} z
						\end{cases}
					\]
					la funzione di trasferimento non cambia:
					\[
						\begin{bmatrix}
							0 & 1
						\end{bmatrix}
						\begin{bmatrix}
							\dfrac{1}{s} & 0
							\\[1em]
							\dfrac{1}{s^2} & \dfrac{1}{s}
						\end{bmatrix}
						\begin{bmatrix}
							1\\
							0
						\end{bmatrix} = \dfrac{1}{s^2}
					\]
				\item 
					\[
						S:
						\begin{cases}
							\dot x =
							\begin{bmatrix}
								0 & 1 & 9\\
								0 & 0 & 0\\
								0 & 0 & \sqrt{2}
							\end{bmatrix} x+
							\begin{bmatrix}
								0\\
								1\\
								0
							\end{bmatrix} u
							\\[1em]
							y =
							\begin{bmatrix}
								1 & 0 & \sqrt{17}
							\end{bmatrix} x
						\end{cases}
					\]
					Si nota che $ x_3 $ non dipende da $ x_1 $ e $ x_2 $ e dal controllo, quindi costituisce la parte non controllabile:
					\[
						\varphi(s) = s^2(s-\sqrt{2}) \quad\text{dove}\quad \sqrt{2} \ \text{\'e l'autovalore non controllabile}
					\]
					Se svolgiamo i calcoli otteniamo sempre:
					\[
						T(s) = \dots = \dfrac{1}{s^2}
					\]
			\end{itemize}
		\end{mdframed}

	\section{Realizzazione minima}
		Si vuole che la realizzazione sia la pi\'u compatta possibile. Si sceglie quindi:
		\[
			n_x = \pDeg{\text{denominatore di}\ T(s)} = \text{numero autovalori controllabili e osservabili}
		\]
		
		Una realizzazione minima \'e sempre completamente controllabile \& osservabile. Allora una qualsiasi realizzazione non minima non pu\'o essere completamente controllabile \& osservabile.
		
		Tutte le realizzazioni minime di una $ T(s) $ sono in relazione tra loro mediante un opportuno cambio di base.
		
	\subsection{Forma Compagna Controllabile}
		Data la funzione di trasferimento di un sistema fisicamente realizzabile:
		\[
			T(s) = \dfrac{b_n s^n + b_{n-1} s^{n-1} + \ldots + b_0}{s^n + a_{n-1} s^{n-1} + \ldots + a_1 s + a_0}
			\qquad
			\begin{aligned}
				\pDeg{Num} \leq n\\
				\pDeg{Denom} = n
			\end{aligned}
		\]
		la forma compagna controllabile \'e:
		\[
			\begin{cases}
				\dot{\vec x} = A_C \vec x + B_C u\\
				y = C_C \vec x + D_C u
			\end{cases}
		\]
		\[
			\begin{aligned}
				\dot{\vec x} &=
				\begin{bmatrix}
					0 & 1 & 0 & \dots & 0\\
					0 & 0 & 1 & \dots & 0\\
					\vdots & \vdots & \vdots & \ddots & \vdots\\
					0 & 0 & 0 & \dots & 1\\
					-a_0 & -a_1 & -a_2 & \dots & -a_{n-1}
				\end{bmatrix} \vec x+
				\begin{bmatrix}
					0\\
					0\\
					\vdots\\
					0\\
					1
				\end{bmatrix} u
				\\
				y &=
				\begin{bmatrix}
					\tilde b_0 & \tilde b_1 & \dots & \tilde b_{n-1}
				\end{bmatrix} \vec x + b_n u
			\end{aligned}
		\]
		Come calcolare i $ \tilde b_i $?
		\[
			T(s) = \dfrac{\bar B(s)}{\bar A(s)} =
		\]
		\[
			\begin{array}{cccc|c}
				b_n s^n & +b_{n-1} s^{n-1} & +\dots & +b_0 & s^n + a_{n-1} s^{n-1} + \ldots + a_1 s + a_0
				\\\cline{5-5}
				b_n s^n & +a_{n-1} b_n s^{n-1} & +\dots & +a_0 b_n & b_n
				\\\cline{1-4}
				0 & +(b_{n-1} - a_{n-1} b_n) s^{n-1} & +\dots & +(b_0 - a_0 b_n)
			\end{array}
		\]
		\newline
		\[
			\Rightarrow\quad T(s) = b_n + \dfrac{(b_{n-1} - a_{n-1} b_n)s^{n-1} + \dots + (b_0 - a_0 b_n)}{s^n + a_{n-1} s^{n-1} + \dots + a_0} = b_n + \dfrac{\tilde b_{n-1} s^{n-1} + \dots + \tilde b_1 s + \tilde b_0}{s^n + a_{n-1} s^{n-1} + \dots + a_0} = b_n + \tilde T(s)
		\]
		dove i coefficienti del resto della divisione sono $ \tilde b_i \triangleq b_i - a_i b_n $
		\[
			Y_f(s) = T(s) U(s) \quad\Rightarrow\quad y_f(t) = b_n u(t) + \ldots
		\]
		$ b_n $ rappresenta la dipendenza istantanea tra l'ingresso e l'uscita.\\
		\newline
		La forma compagna controllabile \'e una realizzazione minima di $ T(s) $.
		
		\begin{mdframed}[style=Exercise]
			\begin{Exercise}[title={Calcolare la forma compagna controllabile}, difficulty=1]
				\[
					T(s) = \dfrac{s^2-1}{s^2+2s+3} \quad\Rightarrow\quad n = \pDeg{\text{Denom}}
				\]
				Si pu\'o calcolare la FCC seguendo due strade:
				\begin{itemize}
					\item 
						calcolandosi gli elementi delle matrici con le formule:
						\[
							\begin{aligned}
								a_0 &= 3\\
								a_1 &= 2
							\end{aligned}
							\qquad
							\begin{aligned}
								b_0 &= -1\\
								b_1 &= 0\\
								b_2 &= 1
							\end{aligned}
							\qquad\Rightarrow\quad
							\begin{aligned}
								\tilde b_0 &= b_0 - a_0 b_2 = -4\\
								\tilde b_1 &= b_1 - a_1 b_2 = -2
							\end{aligned}
						\]
						\[
							\begin{aligned}
								\dot x &= 
								\begin{bmatrix}
									0 & 1\\
									-3 & -2
								\end{bmatrix} x+
								\begin{bmatrix}
									0\\
									1
								\end{bmatrix} u
								\\
								y &=
								\begin{bmatrix}
									-4 & -2
								\end{bmatrix} x + u
							\end{aligned}
						\]
						otteniamo lo stesso risultato: $ \tilde b_0 = -4 $, $ \tilde b_1 = -2 $
					\item 
						eseguendo la divisione tra polinomi:
						\[
							\begin{array}{ccc|c}
								s^2 & & -1 & s^2 + 2s + 3
								\\\cline{4-4}
								s^2 & +2s & +3 & 1
								\\\cline{1-3}
								0 & -2s & -4 &
							\end{array}
						\]
				\end{itemize} 
			\end{Exercise}
		\end{mdframed}

	\subsection{Forma Compagna Osservabile}
		\[
			\Rightarrow\quad\text{FCO}\quad
			\begin{cases}
				\dot{\vec z} = A_C^T \vec z + C_C^T u &= A_O x + B_O u\\
				y = B_C^T \vec z + D_C u &= C_O x + D_O u
			\end{cases}
		\]
		La forma compagna osservabile \'e il \textbf{sistema duale} della forma compagna controllabile.
		\[
			\begin{aligned}
				\dot{\vec x} &=
				\begin{bmatrix}
					0 & 0 & \dots & 0 & -a_0 \\
					1 & 0 & \dots & 0 & -a_1 \\
					0 & 1 & \dots & 0 & -a_2 \\
					\vdots & \vdots & \ddots & \vdots & \vdots \\
					0 & 0 & \dots & 1 & -a_{n-1}
				\end{bmatrix} \vec x+
				\begin{bmatrix}
					\tilde b_0\\
					\tilde b_1\\
					\tilde b_2\\
					\vdots\\
					\tilde b_{n-1}
				\end{bmatrix} u
				\\
				y &=
				\begin{bmatrix}
					0 & 0 & \dots & 0 & 1
				\end{bmatrix} \vec x + b_n u
			\end{aligned}
		\]
		
		\begin{mdframed}[style=Esempio]
			\paragraph{Esempio} integratore
				\[
					T(s) = \dfrac{1}{s} \quad\Rightarrow\quad n=1 \qquad
					a_0 = 0 \qquad
					\begin{aligned}
						b_0 &= 1\\
						b_1 &= 0
					\end{aligned}
				\]
				Quando la $ T(s) $ \'e strettamente propria $ \quad\Rightarrow\quad \tilde b_i \equiv b_i $ perch\'e $ b_n = 0 $.
				\[
					\begin{cases}
						\dot x = [-a_0] x + [b_0] u\\
						y = [1] x + b_n u
					\end{cases}
					\quad\Rightarrow\quad
					\begin{cases}
						\dot x = u\\
						y = x
					\end{cases}
				\]
		\end{mdframed}
	
		\begin{mdframed}[style=Esempio]
			\paragraph{Esempio}
				\[
					T(s) = \dfrac{3}{s+2} = \dfrac{s+2-2-3}{s+2} = 1+\dfrac{-5}{s+2}
				\]
				\[
					\begin{cases}
						\dot x = -2x + u\\
						y = -5x + u
					\end{cases}
				\]
		\end{mdframed}
		
	\subsection{Propriet\'a di FCC e FCO}
		\begin{itemize}
			\item 
				\[
					A_C = A_O^T =
					\begin{bmatrix}
						0 & 1 & \dots & 0\\
						\vdots & \vdots & \ddots & \vdots\\
						0 & 0 & \dots & 1\\
						-a_0 & -a_1 & \dots & -a_{n-1}
					\end{bmatrix}
					\quad\Rightarrow\quad
					\varphi(s) = s^n + a_{n-1} s^{n-1} + \ldots + a_0
				\]
				FCC e FCO non servono solo a realizzare una funzione di trasferimento, ma sono utilizzabili anche per "assegnazione di autovalori".
				
				\begin{mdframed}[style=Esempio]
					\paragraph{Esempio} voglio ottenere gli autovalori $ s_1 = 0, s_2 = -1, s_3 = 1 $, cio\'e
						\[
							\varphi(s) = s(s+1)(s-1) = s(s^2-1) = s^3 - s
						\]
						La forma compagna controllabile di questo sistema sarebbe:
						\[
							\dot x =
							\begin{bmatrix}
								0 & 1 & 0\\
								0 & 0 & 1\\
								0 & 1 & 0
							\end{bmatrix} x+
							\begin{bmatrix}
								0\\
								0\\
								1
							\end{bmatrix} u
						\]
				\end{mdframed}
			\item 
				Supponiamo sia dato un sistema qualsiasi in FCC (non \'e detto che sia ottenuto da una $ T(s) $) al quale applichiamo il teorema di Kalmann di controllabilit\'a:
				\[
					P_C =
					\begin{bmatrix}
						B | AB | \dots | A^{n_x-1}B
					\end{bmatrix} =
					\begin{bmatrix}
						0 		& 0 		& 0 		& \dots 	& \dots 	& 1\\
						\vdots 	& \vdots 	& 0			& 1 		& \dots		& *\\
						\vdots 	& \vdots	& \vdots	& \vdots	& \vdots	& \vdots\\
						0 		& 0 		& 1 		& *		 	& \dots 	& *\\
						1 		& -a_{n_x-1} & * 		& * 	& \dots		& *
					\end{bmatrix}
				\]
				Poich\'e le colonne sono tutte linearmente indipendenti (qualunque valore ci sia al posto degli asterischi), il rango \'e massimo: $ rank \left\lbrace P_C \right\rbrace = n_x $ 
				La coppia $ (A_C, B_C) $ \'e sempre completamente controllabile. Se $ \pDeg{\varphi_{CO}(s)} < n_x $, allora si ha perdita di osservabili\'a.
			\item 
				Un sistema in FCO \'e sempre completamente osservabile. Se  $ \pDeg{\varphi_{CO}(s)} < n_x $, allora si ha perdita di controllabilit\'a.
			\item 
				Se si usano FCC o FCO per una realizzazione minima di una $ T(s) $, il risultato \'e sempre completamente controllabile \& osservabile.
				
				Quindi se avessimo un sistema descritto attraverso un'equazione differenziale lineare di ordine $ n $ a coefficienti costanti (il sistema sarebbe completamente osservabile. vedi \ref{sec:sistema_eq_diff}), potremmo scrivere la $ T(s) $, semplificare i fattori comuni tra $ A(s) $ e $ B(s) $ e scrivere la FCC o la FCO. Cos\'i facendo, per\'o, metteremmo in equazione di stato solo la parte controllabile \& osservabile del sistema, ma non consideriamo il resto (la parte controllabile persa durante le semplificazioni). In base a $ P(s) $ (i fattori in comune tra $ A(s) $ e $ B(s) $) conviene utilizzare una forma compagna rispetto all'altra per :
				\begin{itemize}
					\item 
						se $ P(s) = 1 $, il sistema \'e completamente controllabile \& osservabile, quindi \'e indifferente usare FCC o FCO.
					\item 
						se $ P(s) \neq 1 $, cio\'e ci sono semplificazioni tra numeratore e denominatore della $ T(s) $ che corrispondono a perdite di controllabilit\'a, bisogna usare la FCO (perch\'e l'equazione differenziale \'e sempre completamente osservabile).
				\end{itemize}
		\end{itemize}
\end{document}