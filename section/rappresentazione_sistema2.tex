\documentclass[../main.tex]{subfiles}

\begin{document}
	Un sistema pu\'o essere descritto in diversi modi:
	\begin{itemize}
		\item 
			equazione differenziale lineare a coefficienti costanti
		\item 
			matrice di trasferimento (che descrive solamente la relazione tra ingresso e uscita)
		\item 
			equazione di stato
	\end{itemize}
	
	Finora abbiamo studiato come ottenere la matrice di trasferimento partendo o dall'equazione differenziale o dall'equazione di stato.
	
	\section{Relazioni tra controllabilit\'a/osservabilit\'a e l'equazione differenziale}
		\[
			A(D)y(t) = B(D)u(t)
		\]
		Trasformando con Laplace si ottiene:
		\[
			Y(s) = Y_f(s) + Y_l(s) = T(s)U(s) + Y_l(s) = \dfrac{\bar B(s)}{\bar{\bar{A}}(s)} U(s) + \dfrac{I(s)}{A(s)} U(s)
		\]
		Per i sistemi causali $ I(s) $ \'e un qualsiasi polinomio di grado $ < n = \pDeg{A(s)} $. Quindi in $ Y_l(s) $ possono essere presenti tutti i modi di evoluzione del sistema con le loro molteplicit\'a $ \quad\Rightarrow\quad $ tutti i modi di evoluzione del sistema sono osservabili.
		
		Allora un equazione differenziale lineare a coefficienti costanti \'e \textbf{sempre} completamente osservabile perch\'e $ \varphi(s) = \varphi_O(s) = A(s) $.
		
		Un sistema descritto in questo modo \'e completamente controllabile se $ \varphi_C = \varphi(s) = A(s) $, ma essendo sempre osservabile, deve valere anche  $ \varphi_{CO}(s) = \varphi(s) = A(s) $. Poich\'e $ \varphi_{CO}(s) $ \'e il polinomio caratteristico associato a $ T(s) = \sfrac{\bar B(s)}{\bar A(s)} $ per sistemi SISO, allora $ \varphi_{CO}(s) = \bar A(s) $.
		
		Eventuali semplificazioni tra $ A(s) $ e $ B(s) $ corrispondono a perdite di controllabilit\'a:
		\begin{align*}
			B(s) &= \bar B(s) \cdot P(s)\\
			A(s) &= \underbrace{\bar{A}(s)}_{C} \cdot \underbrace{P(s)}_{NC}
		\end{align*}
		dove $ \bar A(s) $ e $ \bar B(s) $ sono coprimi.
		
		Quindi un sistema \'e completamente controllabile quando: 
		\[
			A(s) = \bar A(s)
		\]
		\newline
		Riepilogo:
		\begin{itemize}
			\item $ \varphi(s) = A(s) $
			\item $ \varphi_{C,NO}(s) = \varphi_{NC,NO}(s) = 1 $ perch\'e il sistema \'e completamente osservabile.
			\item $ \varphi_{CO}(s) = \bar A(s) $
			\item $ \varphi_{NC,O}(s) = P(s) $
		\end{itemize}
\end{document}