\documentclass[../main.tex]{subfiles}

\begin{document}
	\section{Retroazione algebrica sull'uscita}
		Consideriamo il sistema $ S $ dove dall'uscita eliminiamo il termine $ Du $ per eliminare i possibili loop algebrici che si avrebbero con una retroazione:
		\[
			S:
			\begin{cases}
				\dot x = Ax+Bu\\
				y = Cx
			\end{cases}
		\]
		
		Effettuiamo la retroazione $ u = Ky +v $:
		\[
			S^{*}:
			\begin{cases}
				\dot x = Ax + B(Ky+v) = (A+BKC)x + Bv = A^{*}x + Bv\\
				y = Cx
			\end{cases}
		\]
		
		Si pu\'o osservare che:
		\begin{itemize}
			\item 
				la retroazione appena effettuata \'e simile alla retroazione algebrica sullo stato $ u = \hat K x+ v $: infatti la retroazione algebrica sull'uscita \'e una particolare retroazione algebrica sullo stato, ma con pi\'u vincoli.
			\item 
				la retroazione algebrica sullo stato non modifica le propriet\'a di raggiungibilit\'a.
		\end{itemize}
		Grazie a queste due propriet\'a otteniamo che:
		\begin{center}
			\fbox{
				\begin{minipage}{.8\linewidth}
					La retroazione algebrica sull'uscita NON modifica le propriet\'a di controllabilit\'a del sistema
				\end{minipage}
			}
		\end{center}
		Analogamente per l'osservabilit\'a:
		\begin{center}
			\fbox{
				\begin{minipage}{.8\linewidth}
					La retroazione algebrica sull'uscita NON modifica le propriet\'a di osservabilit\'a del sistema
				\end{minipage}
			}
		\end{center}
		\textbf{Dimostriamo} quest'ultima propriet\'a:
		supponiamo che $ \hat{\vec x} \in X_{NO}^{*} $ cio\'e che $ \hat{\vec x} $ sia indistinguibile dallo stato zero dopo la retroazione, dove
		\[
			X_{NO}^{*} = ker(Q^{*}) = ker
			\begin{bmatrix}
				C\\
				CA^{*}\\
				C(A^{*})^2\\
				\vdots\\
				C(A^{*})^{n_x-1}
			\end{bmatrix} =
			\begin{bmatrix}
				C\\
				C(A+BKC)\\
				C(A+BKC)^2\\
				\vdots\\
				C(A+BKC)^{n_x-1}
			\end{bmatrix}
		\]
		quindi per definizione di kernel $ Q^{*} \vec x = 0 $:
		\[
			\begin{bmatrix}
				C\\
				C(A+BKC)\\
				C(A+BKC)^2\\
				\vdots\\
				C(A+BKC)^{n_x-1}
			\end{bmatrix} \hat{\vec x} =
			\begin{bmatrix}
				0\\
				0\\
				0\\
				\vdots\\
				0
			\end{bmatrix}
		\]
		effettuiamo il conto riga per riga:
		\begin{enumerate}
			\item
				$ C \hat{\vec x} = 0 $
			\item
				$ C(A+BKC) \hat{\vec x} = CA \hat{\vec x} + CBK\underbrace{C \hat{\vec x}}_{=0} = 0 \quad\Rightarrow\quad CA \hat{\vec x} = 0 $
			\item 
				$ C(A+BKC)^2 \hat{\vec x} = C(A^2 + ABKC + BKCA + BKCBKC) \hat{\vec x} = $
				\\
				$ = CA^2 \hat{\vec x} + ABK \underbrace{C \hat{\vec x}}_{=0} + BK\underbrace{CA \hat{\vec x}}_{=0} + BKCBK\underbrace{C \hat{\vec x}}_{=0} = CA^2 \hat{\vec x} $
			\item $ \dots $
			\item[n.]
				$ C A^{n_x-1} \hat{\vec x} = 0 $
		\end{enumerate}
		\[
			\Rightarrow\quad
			\begin{bmatrix}
				C\\
				CA\\
				\vdots\\
				CA^{n_x-1}
			\end{bmatrix} 
			= Q \hat{\vec x} = \hat{\vec x} = 0
			\quad\Rightarrow\quad
			\hat{\vec x} \in Ker(Q) = X_{NO}
		\]
		c.v.d.
\end{document}