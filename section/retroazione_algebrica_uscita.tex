\documentclass[../main.tex]{subfiles}

\begin{document}
	\section{Retroazione algebrica sull'uscita}
		Consideriamo il sistema $ S $ dove dall'uscita eliminiamo il termine $ Du $ per evitare i possibili loop algebrici che si avrebbero con una retroazione:
		\[
			S:
			\begin{cases}
				\dot x = Ax+Bu\\
				y = Cx
			\end{cases}
		\]
		
		Effettuiamo la retroazione $ u = Ky +v $:
		\[
			S^{*}:
			\begin{cases}
				\dot x = Ax + B(Ky+v) = (A+BKC)x + Bv = A^{*}x + Bv\\
				y = Cx
			\end{cases}
		\]
		
		Si pu\'o osservare che:
		\begin{itemize}
			\item 
				la retroazione appena effettuata \'e simile alla retroazione algebrica sullo stato $ u = \hat K x+ v $: infatti la retroazione algebrica sull'uscita \'e una particolare retroazione algebrica sullo stato, ma con pi\'u vincoli.
			\item 
				la retroazione algebrica sullo stato non modifica le propriet\'a di raggiungibilit\'a.
		\end{itemize}
		Grazie a queste due propriet\'a otteniamo che:
		\begin{center}
			\fbox{
				\begin{minipage}{.8\linewidth}
					La retroazione algebrica sull'uscita NON modifica le propriet\'a di controllabilit\'a del sistema
				\end{minipage}
			}
		\end{center}
		Analogamente per l'osservabilit\'a:
		\begin{center}
			\fbox{
				\begin{minipage}{.8\linewidth}
					La retroazione algebrica sull'uscita NON modifica le propriet\'a di osservabilit\'a del sistema
				\end{minipage}
			}
		\end{center}
		\textbf{Dimostriamo} quest'ultima propriet\'a:
		supponiamo che $ \hat{\vec x} \in X_{NO}^{*} $ cio\'e che $ \hat{\vec x} $ sia indistinguibile dallo stato zero dopo la retroazione, dove
		\[
			X_{NO}^{*} = ker(Q^{*}) = ker
			\begin{bmatrix}
				C\\
				CA^{*}\\
				C(A^{*})^2\\
				\vdots\\
				C(A^{*})^{n_x-1}
			\end{bmatrix} =
			\begin{bmatrix}
				C\\
				C(A+BKC)\\
				C(A+BKC)^2\\
				\vdots\\
				C(A+BKC)^{n_x-1}
			\end{bmatrix}
		\]
		quindi per definizione di kernel $ Q^{*} \vec x = 0 $:
		\[
			\begin{bmatrix}
				C\\
				C(A+BKC)\\
				C(A+BKC)^2\\
				\vdots\\
				C(A+BKC)^{n_x-1}
			\end{bmatrix} \hat{\vec x} =
			\begin{bmatrix}
				0\\
				0\\
				0\\
				\vdots\\
				0
			\end{bmatrix}
		\]
		effettuiamo il conto riga per riga:
		\begin{enumerate}
			\item
				$ C \hat{\vec x} = 0 $
			\item
				$ C(A+BKC) \hat{\vec x} = CA \hat{\vec x} + CBK\underbrace{C \hat{\vec x}}_{=0} = 0 \quad\Rightarrow\quad CA \hat{\vec x} = 0 $
			\item 
				$ C(A+BKC)^2 \hat{\vec x} = C(A^2 + ABKC + BKCA + BKCBKC) \hat{\vec x} = $
				\\
				$ = CA^2 \hat{\vec x} + ABK \underbrace{C \hat{\vec x}}_{=0} + BK\underbrace{CA \hat{\vec x}}_{=0} + BKCBK\underbrace{C \hat{\vec x}}_{=0} = CA^2 \hat{\vec x} $
			\item $ \dots $
			\item[n.]
				$ C A^{n_x-1} \hat{\vec x} = 0 $
		\end{enumerate}
		\[
			\Rightarrow\quad
			\begin{bmatrix}
				C\\
				CA\\
				\vdots\\
				CA^{n_x-1}
			\end{bmatrix} 
			= Q \hat{\vec x} = \hat{\vec x} = 0
			\quad\Rightarrow\quad
			\hat{\vec x} \in Ker(Q) = X_{NO}
		\]
		c.v.d.
		
	\subsection{Polinomio caratteristico}
		Supponiamo di effettuare la decomposizione di Kalmann al sistema retroazionato:
		\[
			\tilde S^{*}:
			\begin{cases}
				\dot z = (\tilde A + \tilde B K \tilde C)z + \tilde B v = \tilde A^{*} z + \tilde v
				\\
				y = \tilde C z
			\end{cases}
		\]
		\[
			\tilde B K \tilde C =
			\begin{bmatrix}
				\tilde B_1\\
				\tilde B_2\\
				0\\
				0
			\end{bmatrix} K
			\begin{bmatrix}
				0\\
				\tilde C_2\\
				0\\
				\tilde C_4
			\end{bmatrix} =
			\begin{bmatrix}
				0 & \tilde B_1 K \tilde C_2 & 0 & \tilde B_1 K \tilde C_4\\
				0 & \tilde B_2 K \tilde C_2 & 0 & \tilde B_2 K \tilde C_4\\
				0 & 0 & 0 & 0\\
				0 & 0 & 0 & 0
			\end{bmatrix}
		\]
		\[
			\tilde A^{*} =
			\begin{bmatrix}
				\tilde A_{11} & (\tilde A_{12} + \tilde B_1 K \tilde C_2) & \tilde A_{13} & (\tilde A_{14} + \tilde B_1 K \tilde C_4)\\
				0 & (\tilde A_{22} + \tilde B_2 K \tilde C_2) & 0 & (\tilde A_{24} + \tilde B_2 K \tilde C_4)\\
				0 & 0 & \tilde A_{33} & \tilde A_{34}\\
				0 & 0 & 0 & \tilde A_{44}
			\end{bmatrix}
		\]
		
		$ \varphi^{*}(s) $ \'e il polinomio caratteristico associato $ A + BKC $. $ \tilde \varphi^{*}(s) $ \'e il polinomio caratteristico del sistema retroazionato decomposto con Kalmann, associato a $ \tilde A + \tilde B K \tilde C $. Nonostante ci\'o il polinomio caratteristico \'e invariante ai cambiamenti di base:
		\[
			\varphi^{*}(s) = \tilde \varphi^{*}(s)
		\]
		
		Inoltre la matrice $ \tilde A^{*} $ ha mantenuto la struttura triangolare a blocchi:
		\[
			\varphi^{*}(s) = \tilde \varphi^{*}(s) = det(sI-\tilde A_{11}) \cdot det[sI-(\tilde A_{22} + \tilde B_2 K \tilde C_2)] \cdot det(sI-\tilde A_{33}) \cdot det(sI-\tilde A_{44})
		\]
		Rispetto a quanto avevamo trovato nella \hyperref[sec:decomp_kalmann_polinomio]{decomposizione di Kalmann}, solo il secondo fattore \'e stato modificato dalla retroazione:
		\begin{center}
			\fbox{
				\begin{minipage}{.8\linewidth}
					La retroazione algebrica sull'uscita pu\'o modificare solo la parte controllabile e osservabile del sistema.
				\end{minipage}
			}
		\end{center}
		Vengono "modificati" e non "assegnati a piacere" gli autovalori controllabili \textbf{e} osservabili a causa del vincolo imposto dalla matrice $ C $. Se $ rank(C) = n_x $ allora la retroazione sull'uscita equivale alla retroazione algebrica sullo stato.
		
		\begin{Exercise}[title={Retroazione algebrica sull'uscita}, difficulty=1]
			\[
				\begin{cases}
					\dot x=
					\begin{bmatrix}
						0 & 1\\
						0 & 0
					\end{bmatrix} x+
					\begin{bmatrix}
						0\\
						1
					\end{bmatrix} u
					\\
					y =
					\begin{bmatrix}
						1 & 0
					\end{bmatrix} x
				\end{cases}
			\]
			Il sistema ha $ n_x = 2 $, ma la retroazione sull'uscita \'e $ y = Ky + v $ con $ K \in \R $ (scalare): \'e evidente che con la sola variabile $ K $ non posso fissare entrambi gli autovalori del sistema.
			
			Studiamo velocemente il sistema:
			\begin{itemize}
				\item 
					se nella matrice compare un denominatore di grado massimo, allora esso sar\'a il polinomio caratteristico: $ \varphi(s) = s^2 $
				\item 
					poich\'e $ m(s) \subseteq\cdot \varphi(s) $, i possibili polinomi minimi sono: $ s $ e $ s^2 $. Verifichiamo se $ s $ \'e polinomio annullante per $ A $, cio\'e $ m(A) = 0 $. Tuttavia $ A \neq 0 $, $ A $ non \'e una matrice nulla, quindi significa che $ m(s) = s^2 $ (infatti $ A^2 = 0 $).
				\item 
					l'unica radice di $ m(s) $ \'e a $ \Re = 0 $ con molteplicit\'a $ 2 $ quindi il sistema \'e instabile.
				\item 
					controllabilit\'a:
					\[
						(sI-A)^{-1}B=
						\begin{bmatrix}
							\dfrac{1}{s} & \dfrac{1}{s^2}\\
							0 & \dfrac{1}{s}
						\end{bmatrix}
						\begin{bmatrix}
							0\\
							1
						\end{bmatrix} =
						\begin{bmatrix}
							\dfrac{1}{s^2}\\
							\dfrac{1}{s}
						\end{bmatrix}
					\]
					calcoliamo velocemente $ \varphi_C(s) = s^2 \equiv m_C(s) $.\\
					Oppure col teorema di Kalmann:
					\[
						P =
						\begin{bmatrix}
							0 & 1\\
							1 & 0
						\end{bmatrix}
					\]
					Il sistema \'e completamente controllabile.
				\item 
					osservabilit\'a:
					\[
						C(sI-A)^{-1} = 
						\begin{bmatrix}
							\dfrac{1}{s} & \dfrac{1}{s^2}
						\end{bmatrix}
						\quad\Rightarrow\quad \varphi_O(s) = s^2
					\]
					oppure
					\[
						Q =
						\begin{bmatrix}
							1 & 0\\
							0 & 1
						\end{bmatrix}
					\]
					il sistema \'e completamente osservabile.
			\end{itemize}
			\begin{enumerate}
				\item 
					\'E possibile rendere il sistema asintoticamente stabile con una retroazione algebrica sull'uscita?\\
					Per vedere ci\'o calcoliamo il polinomio caratteristico del sistema retroazionato:
					\[
						A^{*}=(A+BKC)
						\begin{bmatrix}
							0 & 1\\
							0 & 0
						\end{bmatrix} +
						\begin{bmatrix}
							0\\
							1
						\end{bmatrix} K
						\begin{bmatrix}
							1 & 0
						\end{bmatrix} =
						\begin{bmatrix}
							0 & 1\\
							0 & 0
						\end{bmatrix} +
						\begin{bmatrix}
							0 & 0\\
							K & 0
						\end{bmatrix} =
						\begin{bmatrix}
							0 & 1\\
							K & 0
						\end{bmatrix}
					\]
					\[
						\varphi^{*}(s) = det(sI-A^{*}) = det
						\begin{bmatrix}
							s & -1\\
							-K & s
						\end{bmatrix} = s^2 - K
					\]
					per $ K \geq 0 $ il sistema resta instabile, mentre per $ K < 0 $ \'e semplicemente stabile.
				\item 
					Con la retroazione \'e possibile assegnare a piacere qualche autovalore?
					S\'i purhc\'e i coefficienti del polinomio che vogliamo ottenere siano reali.
			\end{enumerate}
		\end{Exercise}
	
	\subsection{Stabilit\'a BIBO}
		\[
			\text{stabilit\'a BIBO}\quad\Rightarrow\quad \varphi_{CO}(s)\ \text{non ha radici a}\ \Re > 0
		\]
		$ \varphi_{CO}(s) $ \'e il polinomio caratteristico associato a $ T(s) $ prima della retroazione. Se si pu\'o calcolare $ T^{*}(s) = T_{yv}(s) $ (dopo la retroazione), allora $ \varphi^{*0}_{CO}(s) $ sar\'a il polinomio caratteristico associato a $ T^{*}(s) $.
		\begin{Exercise}[title={Studiare retroazione algebrica uscita attraverso l'algebra dei blocchi}]
			Riprendiamo l'esercizio precedente e calcoliamo la funzione di trasferimento:
			\[
				C(sI-A)^{-1}B +D =
				\begin{bmatrix}
					1 & 0
				\end{bmatrix}
				\begin{bmatrix}
					\dfrac{1}{s} & \dfrac{1}{s^2}\\
					0 & \dfrac{1}{s}
				\end{bmatrix}
				\begin{bmatrix}
					0\\
					1
				\end{bmatrix} =
				\begin{bmatrix}
					1 & 0
				\end{bmatrix}
				\begin{bmatrix}
					\dfrac{1}{s^2}\\
					\dfrac{1}{s}
				\end{bmatrix} =
				\dfrac{1}{s^2}
			\]
			Applichiamo la retroazione algebrica sull'uscita:
			\[
				T_{yv}(s) = \dfrac{\dfrac{1}{s^2}}{1-K\dfrac{1}{s^2}} = \dfrac{\dfrac{1}{s^2}}{\dfrac{s^2-K}{s^2}} = \dfrac{1}{s^2 - K}
			\]
			\[
				\Rightarrow\quad \varphi^{*}_{CO}(s) = s^2 -K
			\]
			Abbiamo ottenuto molto pi\'u facilmente lo stesso risultato di prima.
		\end{Exercise}
	
		\begin{Exercise}[title={Studiare stabilit\'a BIBO dopo retroazione sull'uscita}, difficulty=1]
			\[
				T(s) = \dfrac{1}{s(s-1)}
			\]
			si tratta di un sistema non stabile BIBO perch\'e i poli di $ T(s) $ sono $ 0 $ e $ 1 $ (una radice ha $ \Re > 0 $). Con la retroazione sull'uscita si pu\' rendere il sistema stabile BIBO?
			\[
				T^{*}(s) = \dfrac{\dfrac{1}{s(s+1)}}{1-\dfrac{K}{s(s-1)}} = \dfrac{1}{s^2-s-K}
			\]
			$ \exists K $ tale che i poli di $ T^{*}(s) $ siano tutti a $ \Re < 0 $?\\
			Con la regola di Cartesio si trova subito una variazione (tra $ s^2 $ e $ -s $) che determina la presenza di almeno una radice a $ \Re > 0 $. Quindi il sistema retroazionato non \'e stabile BIBO per nessun $ K $. Analizziamo comunque i casi al variare di $ K $:
			\begin{itemize}
				\item $ K < 0 $: due radici a $ \Re > 0 $
				\item $ K > 0 $: una radice a $ \Re > 0 $ e una radice a $ \Re < 0 $
				\item $ K = 0 $: una radice a $ \Re > 0 $ e una radice a $ \Re = 0 $
			\end{itemize}
		\end{Exercise}
\end{document}