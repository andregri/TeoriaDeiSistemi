%using varii
\usepackage{tikz}
\usepackage{circuitikz}
\usetikzlibrary{shapes,arrows,matrix,positioning}
\tikzstyle{block} = [draw, fill=white, rectangle, minimum height=3em, minimum width=6em]
\tikzstyle{sum} = [draw, fill=white, circle]

%____________________________________________
\usepackage{amsmath, amssymb, mathtools}
\usepackage[english,italian]{babel} % per cambiare la lingua principale
\usepackage{pgfplots}
\usepackage{caption}
\usepackage{subcaption}
\usepackage{stackrel}
\usepackage{array}
\usepackage{verbatim}
\usepackage{wrapfig}
\usepackage{exercise}
\usepackage{xfrac}
\usepackage{xcolor} % Required to specify font color

%____________________________________________
\renewcommand{\thepage}{\thechapter.\arabic{page}}% Page numbering style

%____________________________________________

\usepackage[framemethod=TikZ]{mdframed}
\mdfdefinestyle{Exercise}{%
	linecolor=blue,
	outerlinewidth=0.8pt,
	roundcorner=20pt,
	innertopmargin=\baselineskip,
	innerbottommargin=\baselineskip,
	innerrightmargin=20pt,
	innerleftmargin=20pt}
%
\mdfdefinestyle{Esempio}{%
	linecolor=red,
	outerlinewidth=0.8pt,
	roundcorner=20pt,
	innerbottommargin=\baselineskip,
	innerrightmargin=20pt,
	innerleftmargin=20pt}

%____________________________________________
\usepackage{float}
% If you really want a figure to be placed “HERE”, and not where LaTeX wants to put it, use the [H] option of the “float” package which basically turns the floating figure into a regular non-float.
%____________________________________________

%dimensioni pagina
%VECCHIO:  \usepackage[a4paper, total={6in, 9in}]{geometry}
\usepackage{geometry}
\geometry{a4paper,top=2cm,bottom=2cm,left=2.5cm,right=2.5cm,heightrounded,bindingoffset=5mm}
\addtolength{\skip\footins}{20pt}
\setlength{\parskip}{0.2cm}
%____________________________________________

%intestazione e piè pagina
\usepackage{fancyhdr}
\pagestyle {fancy}
\fancyhead [L] {\slshape \rightmark}
\fancyhead [R] {}
\renewcommand{\headrulewidth}{0pt} % no riga in cima
%____________________________________________

\usepackage{amsthm}
\theoremstyle{definition}
\newtheorem{definition}{Definizione}
%____________________________________________

\usepackage{hyperref}
\hypersetup{
	colorlinks=true,
	linkcolor=blue,
	filecolor=magenta,      
	urlcolor=cyan,
	bookmarks=true
}
%____________________________________________