%using varii
\usepackage{tikz}
\usetikzlibrary{arrows,matrix,positioning}
\usepackage{amsmath, amssymb, mathtools}
\usepackage[english,italian]{babel} % per cambiare la lingua principale
\usepackage{pgfplots}
\usepackage{caption}
\usepackage{subcaption}
\usepackage{stackrel}
\usepackage{array}
\usepackage{verbatim}
\usepackage{wrapfig}
\usepackage{exercise}
\usepackage{circuitikz}
\usepackage{xfrac}
\usepackage{xcolor} % Required to specify font color
%____________________________________________

%dimensioni pagina
%VECCHIO:  \usepackage[a4paper, total={6in, 9in}]{geometry}
\usepackage{geometry}
\geometry{a4paper,top=2cm,bottom=2.5cm,left=2.5cm,right=2.5cm,heightrounded,bindingoffset=5mm}
%____________________________________________

%intestazione e piè pagina
\usepackage{fancyhdr}
\pagestyle {fancy}
\fancyhead [L] {\slshape \rightmark}
\fancyhead [R] {}
\renewcommand{\headrulewidth}{0pt} % no riga in cima
%____________________________________________

\usepackage{amsthm}
\theoremstyle{definition}
\newtheorem{definition}{Definizione}
%____________________________________________

\usepackage{hyperref}
\hypersetup{
	colorlinks=true,
	linkcolor=blue,
	filecolor=magenta,      
	urlcolor=cyan,
	bookmarks=true
}
%____________________________________________
%indice all'inizio di ogni sezione
\usepackage{minitoc}
% Commands for a minitoc:
\setcounter{minitocdepth}{3} 
\setlength{\mtcindent}{1pt} 
\renewcommand{\mtifont}{\LARGE \rm} % dimensione font Titolo minitoc
\renewcommand{\mtctitle}{Argomenti trattati:\\} %modifica Titolo minitoc
\renewcommand{\mtcfont}{\normalsize \rm} % dimensione font minitoc
\renewcommand{\mtcSfont}{\normalsize \rm}  % dimensione font Section
\renewcommand{\mtcSSfont}{\normalsize \rm}  % dimensione font SubSection
\renewcommand{\mtcPfont}{\normalsize \rm}  % dimensione font Paragraph
\renewcommand{\mtcSPfont}{\normalsize \rm}  % dimensione font SubParagraph
\nomtcrule %senza righe orizzontali
\nomtcpagenumbers %senza numero di pagina
%____________________________________________